\documentclass{article}
\usepackage{graphicx} % Required for inserting images

\title{Capstone Spring 2023:JPX Tokyo Stock Exchange Prediction}
\author{Assignment: Project Introduction} 
\date{February  2023}

\begin{document}
\maketitle


\section{Background}
Success in any financial market requires one to identify solid investments. When a stock or derivative is undervalued, it makes sense to buy. If it's overvalued, perhaps it's time to sell. While these finance decisions were historically made manually by professionals, technology has ushered in new opportunities for retail investors. Data scientists, specifically, may be interested to explore quantitative trading, where decisions are executed programmatically based on predictions from trained models.

\section{What is the problem?}
There are plenty of existing quantitative trading efforts used to analyze financial markets and formulate investment strategies. To create and execute such a strategy requires both historical and real-time data, which is difficult to obtain especially for retail investors. This project will provide financial data for the Japanese market, allowing retail investors to analyze the market to the fullest extent.


\section{Why is it interesting and important?}
There are plenty of reasons for predicting stock exchanges. This is an important field because it has the potential to drive financial gains, mitigate risks, stabilize the economy and advance technological deviance gains.

One of the primary reasons people are interested in predicting stock prices is for financial gain. Accurately predicting the movement of stocks can help investors make better decisions and increase their profits.

It also helps with risk management. Accurately predicting the stock market can also help investors manage risk by allowing them to adjust their portfolios based on market conditions. This can help prevent losses and stabilize investments.

It is needed for economic stability. The stock market is often seen as an indicator of the overall health of the economy. Accurately predicting market movements can help policymakers and economists make better decisions and take actions to prevent economic instability.

Predicting the stock market requires the use of advanced technologies and techniques, such as machine learning and artificial intelligence. This has led to significant advancements in these fields, which have potential applications beyond stock market prediction. 


\section{Why is it hard? }
The stock market is a complex system with many variables, including economic indicators, political events, and company-specific factors. It can be difficult to accurately capture and model all of these factors, making it challenging to predict future market movements.

The stock market is also inherently unpredictable, with a large degree of randomness and volatility. Even small changes in market sentiment or investor behavior can lead to significant price swings, making it challenging to predict future price movements.

Historical data can be a valuable tool for predicting future stock prices, but the data can be limited, and it may not accurately reflect future market conditions. Additionally, unexpected events can occur that can't be predicted based on past data.

Stock market movements are also influenced by human behavior, including emotions such as fear, greed, and panic. These factors can be challenging to predict and can cause prices to deviate from what might be expected based on fundamental analysis or historical trends.

Overall, stock market prediction is a challenging task due to the complexity of the market, randomness, limited data, and human behavior. While there have been advances in technology and techniques for predicting the stock market, accurate predictions remain difficult to achieve.


\section{What is your plan?}
With the given datasets, my goal is to predict the direction of the stock exchange. I will then choose a machine learning algorithm either linear regression, decision trees, random forests, or neural networks. I also have to split data into training and validation sets. Use the training set to train the model and the validation set to evaluate its performance. And then train and validate the model, fine-tune the model and then test the model. 

Stock exchange prediction is a difficult task, and no model can predict the stock market with 100 percent accuracy. However, with careful data preprocessing, algorithm selection, and model training, I can create a model that can make reasonably accurate predictions. 

Here are the datasets provided by Kaggle:
stock\_prices.csv The core file of interest. Includes the daily closing price for each stock and the target column.

options.csv Data on the status of a variety of options based on the broader market. Many options include implicit predictions of the future price of the stock market and so may be of interest even though the options are not scored directly.

secondary\_stock\_prices.csv The core dataset contains on the 2,000 most commonly traded equities but many fewer liquid securities are also traded on the Tokyo market. 
This file contains data for those securities, which aren't scored but may be of interest for assessing the market as a whole.

trades.csv Aggregated summary of trading volumes from the previous business week.

financials.csv Results from quarterly earnings reports.

stock\_list.csv - Mapping between the SecuritiesCode and company names, plus general information about which industry the company is in.

https://www.kaggle.com/competitions/jpx-tokyo-stock-exchange-prediction/data



\section{What are the performance metrics?}
I don’t have a decided performance metric, but I will choose from some common performance metrics that are used in stock market prediction. 

Mean Squared Error (MSE): This metric measures the average squared difference between the predicted values and the actual values. It is a common metric for regression problems, where the goal is to predict a continuous value, such as stock prices.

Root Mean Squared Error (RMSE): This metric is like MSE, but the square root is taken to make the units of the error the same as the original values.
Mean Absolute Error (MAE): This metric measures the average absolute difference between the predicted values and the actual values. It is another common metric for regression problems.

Directional Accuracy (DA): This metric measures the percentage of predictions that are correct in terms of the direction of the market movement (up or down).

\section{GitHub Repository }
https://github.com/ssarowar/Stock-Exchange-Prediction

\section{Literature Review}

\subsection{Table to compare ML methods}

\begin{tabular}{|l||*{8}{c|}}\hline
\backslashbox{Paper}{ML Methods}
&\makebox[2em]{RA}&\makebox[2em]{ANNs}&\makebox[2em]{SVMS}&\makebox[2em]{DT}&\makebox[2em]{En}&\makebox[2em]{CNN}&\makebox[2em]{RNN}&\makebox[2em]{LSTM}\\\hline\hline
hegazy2014machine & yes & yes & yes & yes & yes & no & no & no\\\hline
CHONG2017187 &no&no&no&no&no&yes&yes&no\\\hline
JIANG2021115537 &no&no&no&no&no&yes&no&yes\\\hline
8541310 &no&no&no&no&no&yes&no&no\\\hline
arXiv:1605.00003 &no&no&no&yes&no&no&no&no\\\hline
This model  &no&no&no&yes&no&no&no&no\\\hline

\end{tabular}

\subsection{hegazy2014machine}
The paper provides a comprehensive review of the state-of-the-art machine learning techniques that have been used for stock prediction. It discusses the importance of accurate stock prediction for investors and traders and highlights the potential benefits of using machine learning techniques to improve the accuracy of these predictions.

The paper presents an overview of several machine learning techniques, including artificial neural networks, support vector machines, decision trees, and genetic algorithms, that have been applied to stock prediction. It provides a detailed explanation of each technique, its advantages and limitations, and its application to stock prediction.

The authors also discuss the factors that affect stock prices, such as market trends, company financials, and macroeconomic indicators. They highlight the importance of incorporating these factors into machine learning models for accurate stock prediction.

The paper concludes with a discussion of the challenges and future directions of machine learning for stock prediction. It notes that accurate stock prediction is a complex problem due to the unpredictability of the stock market, and that further research is needed to develop more accurate and robust machine learning models for stock prediction.

Overall, the paper is a valuable resource for researchers, practitioners, and investors interested in using machine learning techniques for stock prediction. It provides a comprehensive survey of the different techniques available, and highlights the potential benefits and challenges of using these techniques for stock prediction.



\subsection{CHONG2017187}
The paper "Deep learning networks for stock market analysis and prediction: Methodology, data representations, and case studies" provides a comprehensive overview of the application of deep learning networks for stock market analysis and prediction. The authors discuss the challenges and opportunities of using deep learning networks in this field, including data representation, feature extraction, and model selection.

The paper presents several case studies that apply deep learning networks to various aspects of stock market analysis, such as stock price prediction, trend analysis, and sentiment analysis. The authors also discuss different types of deep learning networks, including feedforward neural networks, convolutional neural networks, and recurrent neural networks.

The authors emphasize the importance of selecting appropriate input data and designing effective network architectures for achieving accurate and reliable stock market predictions. They also highlight the potential benefits of combining multiple deep learning models and integrating external data sources, such as news and social media feeds, into the analysis.

Overall, the paper provides a valuable resource for researchers and practitioners interested in applying deep learning networks to stock market analysis and prediction.


\subsection{JIANG2021115537}
The paper "Applications of deep learning in stock market prediction: Recent progress" provides an overview of the recent advancements in using deep learning techniques for predicting stock prices. The author discusses the challenges and opportunities of using deep learning algorithms in stock market prediction, including feature selection, model selection, and the impact of external factors.

The paper covers a range of deep learning models, including feedforward neural networks, recurrent neural networks, convolutional neural networks, and deep belief networks. The author also discusses different approaches to data preprocessing and feature engineering, as well as the use of transfer learning and ensemble methods for improving prediction accuracy.

The author highlights several case studies that demonstrate the effectiveness of deep learning models in predicting stock prices, including studies that use technical indicators, news articles, and social media data as input features. The paper also identifies some of the limitations and challenges of using deep learning models in this field, including the difficulty of interpreting the results and the potential risks associated with relying on machine learning models for financial decision-making.

Overall, the paper provides a valuable overview of the recent progress and future directions of applying deep learning techniques for stock market prediction.


\subsection{8541310}
The paper "Deep Learning for Stock Market Prediction Using Event Embedding and Technical Indicators" proposes a deep learning framework for stock market prediction that incorporates event embedding and technical indicators. The authors highlight the limitations of traditional technical analysis methods and argue that incorporating news events can provide additional insights into stock price movements.

The proposed framework uses a combination of long short-term memory (LSTM) and convolutional neural network (CNN) models to extract features from the input data. The input data includes historical stock prices, technical indicators, and news events represented as word embeddings. The authors also use feature selection techniques to identify the most important features for predicting stock prices.

The authors evaluate the performance of the proposed framework on several real-world datasets and compare it to other machine learning models, such as random forest and support vector regression. The results show that the proposed framework outperforms the other models in terms of prediction accuracy, especially when incorporating news events.

Overall, the paper presents a promising approach for using deep learning techniques to incorporate news events into stock market prediction models, which can provide additional insights and improve prediction accuracy.


\subsection{arXiv:1605.00003}
The article discusses the use of a machine learning algorithm called random forest to predict the direction of stock market prices. The authors explain the concept of random forest and its advantages over other machine learning algorithms. They also describe the dataset used in the study and the various features that were considered for the prediction.

The article goes on to explain the methodology used to train the random forest model and the metrics used to evaluate its performance. The results of the study show that the random forest model outperforms other traditional methods in predicting stock prices.

The authors conclude that the use of machine learning algorithms such as random forest can be an effective tool for predicting stock market trends, and can help investors make informed decisions.

\subsection{Project Work}
A decision tree is a type of machine learning model that uses a tree-like structure to represent decisions and their possible consequences. The tree is constructed by recursively splitting the data into subsets based on the values of input features. At each node of the tree, a decision is made based on the value of a specific feature, and the data is split into subsets accordingly. This process is repeated until a stopping criterion is met, such as a maximum tree depth or a minimum number of data points at a leaf node.

Once we have our labeled data-set and input features, we can use a decision tree algorithm to train a model to predict the target variable. During training, the decision tree algorithm would recursively split the data into subsets based on the input features, with the goal of maximizing the information gain or minimizing the impurity of the subsets.


\section{References}

\cite{hegazy2014machine}

\cite{CHONG2017187}

\cite{JIANG2021115537}

\cite{8541310}

\cite{arXiv:1605.00003}

\cite{JPX} 
JPX Tokyo Stock Exchange Prediction. Kaggle. (n.d.). Retrieved February 16, 2023, from https://www.kaggle.com/competitions/jpx-tokyo-stock-exchange-prediction/data 

\cite{Biswal} 
Biswal, A. (2023, February 16). Stock price prediction using machine learning: An easy guide: Simplilearn. Simplilearn.com. Retrieved February 16, 2023, from https://www.simplilearn.com/tutorials/machine-learning-tutorial/stock-price-prediction-using-machine-learning 

\cite{Yates} 
Yates, T. (2022, July 13). 4 ways to predict market performance. Investopedia. Retrieved February 16, 2023, from 

https://www.investopedia.com/articles/07/mean_reversion_martingale.asp 

\cite{Agrawal} 
Agrawal, R. (2022, July 21). Evaluation metrics for your regression model. Analytics Vidhya. Retrieved February 16, 2023, from 

https://www.analyticsvidhya.com/blog/2021/05/know-the-best-evaluation-metrics-for-your-regression-model/ 


\end{document}

